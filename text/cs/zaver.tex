\chapter*{Závěr}
\addcontentsline{toc}{chapter}{Závěr}

Jednotlivé kroky analýzy videozáznamu skoku o~tyči, mezi které patří extrakce modelu atleta, detekce důležitých momentů skoku, vyhodnocení a~zobrazení parametrů se mi podařilo implementovat. Ačkoliv není přesnost analýzy vždy dostatečná, jedná se o~úvodní krok k~realizaci aplikace, která by videozáznamy dokázala spolehlivě a~přesně analyzovat a~na základě výsledků těchto analýz skoky porovnávat, což byl můj prvotní cíl.

Původně jsem měl v~úmyslu vytvořit aplikaci pro mobilní zařízení, která by videa analyzovala v~průběhu tréninků a~soutěží. Převod aplikace z~počítače na platformu iOS byl složitější, než jsem očekával a~raději jsem svou pozornost zaměřil na vylepšení funkcionality počítačové aplikace. Posléze se ukázalo, že časová náročnost analýzy by bez modifikací implementovaného algoritmu nezvládla videa analyzovat včas.

Značnou pozornost jsem při psaní práce věnoval automatickému nalezení atleta ve videu. Parametry pro vyhledávání postav ve videu se mi bohužel nepodařilo nastavit tak, abych byl s~úspěšností vyhledávání spokojen. Proto je manuální nalezení atleta ve videu výchozím nastavením programu.

Způsob trasování atleta jsem v~průběhu psaní práce několikrát měnil, úspěšnost se stále zlepšovala, ale ani aktuální metoda není $100\%$ úspěšná ve fázi skoku.

Detekce částí atleta má problémy s~malými postavami a~při skoku, především ve fázi skoku, kdy má atlet nohy blízko trupu. Tyto nepřesnosti jsou důsledkem zvoleného modelu, který jednotlivé části těla detekuje a~jeho trénovací databází. Ani pokusy o~natočení snímku detekci příliš nepomáhají.

Na přesnostech detekce atletova těla závisí hodnoty zkoumaných parametrů (a~také důležitých momentů skoku). Například detekce momentu odrazu spoléhá na kvalitní detekci atletových nohou, která ne vždy proběhne. Kvalitní analýza parametrů si zakládá na důkladné znalosti biomechanických principů a~její implementace vyžaduje mezioborovou spolupráci.

Věřím, že jsem svou prací ukázal, že je možné vytvořit program schopný kvalitní analýzy parametrů skoku o~tyči, kterou by bylo možné využívat jak v~tréninkovém, tak závodním prostředí.


















































