\chapter*{Závěr}
\addcontentsline{toc}{chapter}{Závěr}

Jednotlivé kroky analýzy videozáznamu skoku o~tyči, mezi které patří extrakce modelu atleta, detekce důležitých momentů skoku, vyhodnocení a~zobrazení parametrů se mi podařilo implementovat. Ačkoliv není přesnost analýzy vždy dostatečná, jedná se o~úvodní krok k~realizaci aplikace, která by videozáznamy dokázala spolehlivě a~přesně analyzovat a~na základě výsledků těchto analýz skoky porovnávat, což byl můj prvotní cíl.

Původně jsem měl v~úmyslu vytvořit aplikaci pro mobilní zařízení, která by videa analyzovala v~průběhu tréninků a~soutěží. Převod aplikace z~počítače na platformu iOS byl složitější, než jsem očekával a~raději jsem svou pozornost zaměřil na vylepšení funkcionality počítačové aplikace. Posléze se ukázalo, že časová náročnost analýzy by bez modifikací implementovaného algoritmu nezvládla videa analyzovat včas.

Značnou pozornost jsem při psaní práce věnoval automatickému nalezení atleta ve videu. Parametry pro vyhledávání postav ve videu se mi bohužel nepodařilo nastavit tak, abych byl s~úspěšností vyhledávání spokojen. Proto je manuální nalezení atleta ve videu výchozím nastavením programu.

Způsob trasování atleta jsem v~průběhu psaní práce několikrát měnil, úspěšnost se stále zlepšovala, ale ani aktuální metoda není $100\%$ úspěšná ve fázi skoku.

Detekce částí atleta má problémy s~malými postavami a~při skoku, především ve fázi skoku, kdy má atlet nohy blízko trupu. Tyto nepřesnosti jsou důsledkem zvoleného modelu, který jednotlivé části těla detekuje a~jeho trénovací databází. Ani pokusy o~natočení snímku detekci příliš nepomáhají.

Na přesnostech detekce atletova těla závisí hodnoty zkoumaných parametrů (a~také důležitých momentů skoku). Například detekce momentu odrazu spoléhá na kvalitní detekci atletových nohou, která ne vždy proběhne. Kvalitní analýza parametrů si zakládá na důkladné znalosti biomechanických principů a~její implementace vyžaduje mezioborovou spolupráci.

Věřím, že jsem svou prací ukázal, že je možné vytvořit program schopný kvalitní analýzy parametrů skoku o~tyči, kterou by bylo možné využívat jak v~tréninkovém, tak závodním prostředí.




\section*{Navrhovaná vylepšení}
\addcontentsline{toc}{section}{Navrhovaná vylepšení}

Na závěr bych rád popsal, jakými způsoby by bylo možné funkcionalitu programu vylepšit.



\subsection*{Trasování těla}

Úspěšnosti trasování těla atleta při skoku by mohlo pomoci její natáčení podle fáze, ve které se skokan nachází. Přesnějším způsobem trasování může být kombinace implementovaného přístupu s~využitím mřížky a~metody otáčení snímku, kterou program používal předtím. Mřížku by bylo možné otáčet podle úhlu naklonění trupu.



\subsection*{Detekce kostry}

Ideálním přístupem k~detekci pozice těla pro takto specifické využití je natrénování detektoru na vhodné množině dat. K~detekci postavy pouze při skoku o~tyči by většinu takové množiny dat měly tvořit právě snímky zachycující skok o~tyči. Databáze MPII Human Pose dataset \citep{MPIIHPE}, na které je natrénovaný model, který aplikace nyní využívá obsahuje naprosté minimum dat týkajících se skoku o~tyči, aby dobře generalizoval. Pro účely mé aplikace není vysoká míra generalizace nutná, při detekci těla atleta, když se ve snímku poblíž nachází jiná postava přesnosti spíše škodí.

Použití vhodné trénovací databáze by mohlo omezit potřebu určení pozice atleta ve videu pomocí trackerů.

Kromě trénování specializovaného modelu by mohlo být užitečné zkoumat celý výstup sítě, která detekci zajišťuje. Tento způsob by mohl omezit případy, kdy se detekuje více postav najednou (viz sekce \ref{sec:postavy}), ale nepomohla by s~neúspěšnými detekcemi ve fázi skoku.



\subsection*{Časová náročnost}

Rozdělení videa do několika částí a~jejch následné paralelní zpracování by časovou náročnost analýzy značně snížilo. Dalším přístupem by mohlo být rozdělení detekce kostry do více částí a~jejich paralelní zpracování.

Kromě využití paralelismu by mohl pomoci specializovaný model pro detekci částí těla, jelikož nyní při neúspěšné detekci zkouším další natočení snímku, která v~určitých případech nepomohou. Tento model by žádné další natočení nepotřeboval, a~tudíž by se detekce spouštěla na menším množství vstupů.

Časovou náročnost při detekci postav zbytečně zvyšuje množství instancí třídy \texttt{background\_tracker}, díky nimž určuji pohyb postav ve videu. Tyto trackery často trasují stejnou část pozadí, ale aktualizují se pro každou instanci. Původně jsem tracker pozadí implementoval v~závislosti na poloze příslušné postavy a~objektový návrh jsem poté nestihl změnit.


















































