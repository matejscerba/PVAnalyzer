\chapter{Realizace}

\section{Návrh řešení}

\subsection{Zpracování videa}

Ke zpracování videa využiji třídu \texttt{cv::VideoCapture}, která otevře video a~uloží jednotlivé snímky videa. Následně budu zpracovávat video prostřednictvím těchto snímků, aby video nebylo otevřené příliš dlouho.

\subsection{Nalezení atletova těla ve videu}

Prvním krokem k~analýze videozáznamu je nalezení pozice atletova těla ve videu. Pro její určení stačí znát pozici ohraničujícího rámečku atletova těla ve snímku, v~němž je poprvé vidět. Může se stát, že v~prvním snímku videa atlet vidět není, a tedy je nutné specifikovat i~číslo snímku, ve kterém atletův ohraničující rámeček znám.

K~zisku těchto informací lze přistoupit dvěma metodami. První z~nich je manuální zadání ohraničujícího rámečku uživatelem. Pro jednoduchst práce s~programem je vhodné použít grafické rozhraní, které dovolí uživateli definovat rámeček pohybem myši po snímku videa. K~tomuto účelu se výborně hodí metoda \texttt{cv::selectROI}.

Druhou metodou nalezení atletovy pozice ve videu je automatická detekce. K~její realizaci nejprve detekuji ohraničující rámečky postav ve snímku videa a~následně budu sledovat jejich pohyb.

Pro detekci postav ve snímku lze použít strukturu \texttt{cv::HOGDescriptor} a~její metodu \texttt{detectMultiScale}, která je schopná detekovat několik postav různých velikostí v~daném snímku. K~detekci postav ve snímku využívá HOG a~SVM.

Detekované postavy je potřeba trasovat i v~následujících snímcích. K~tomu využiji třídu \texttt{cv::Tracker}.

Jelikož je mým cílem detekce atleta, stačí sledovat jen postavy, které se hýbou. Pohyb postavy určím podle vzájemného pohybu ohraničujícího rámečku postavy a~pozadí snímku. Pohyb pozadí ve videu budu trasovat stejně jako pohyb postav. Vyberu si část pozadí a~její pohyb získám s~použitím třídy \texttt{cv::Tracker}.

Atleta odliším od ostatních pohybujících se postav tak, že provede odraz, tedy se směr jeho pohybu změní směrem vzhůru. Jakmile k~odrazu dojde, zapamatuji si číslo snímku, v~němž jsem atleta poprvé detekoval, a~pozici jeho ohraničujícího rámečku v daném snímku.

\subsection{Detekce kostry atleta}

Pro detekci atletovy kostry využiji natrénovaný model projektu OpenPose\footnote{https://github.com/CMU-Perceptual-Computing-Lab/openpose}. Model je natrénovaný na množině MPII Human Pose Dataset \citep{MPIIHPE}, která obsahuje 25~tisíc obrázků s~více než 40~tisíci postavami s~anotovanými částmi těla. Množina obsahuje postavy vykonávající 410~aktivit, které jsou u~daného obrázku specifikované.

Počet částí těla, které model specifikuje je 16, ale poslední z~nich je pozadí, které budu ignorovat. Tedy se budu zajímat pouze o~prvních 15 částí těla. Jedná se o
\begin{enumerate}
\setcounter{enumi}{-1}
\item hlavu,
\item krk,
\item pravé rameno,
\item pravý loket,
\item pravé zápěstí,
\item levé rameno,
\item levý loket,
\item levé zápěstí,
\item pravou kyčel,
\item pravé koleno,
\item pravý kotník
\item levou kyčel,
\item levé koleno,
\item levý kotník a
\item hrudník.
\end{enumerate}

Model následně načtu do hluboké neuronové sítě \texttt{cv::dnn::Net} pomocí metody \texttt{cv::dnn::readNetFromCaffe}, jelikož je model uložený ve formátu používaného frameworkem Caffe \citep{Caffe}.

Abych mohl síti předat obrázek, je nutné ho nejprve překonvertovat metodou \texttt{cv::dnn::blobFromImage}. Výsledek této konverze je validní vstup pro síť. Jedná se o~čtyřdimenzionální instanci \texttt{cv::Mat}, jejíž rozměry jsou
\begin{itemize}
\item počet vstupních obrázků,
\item počet kanálů vstupních obrázků,
\item výška obrázků a
\item šířka obrázků.
\end{itemize}

Výsledná instance \texttt{cv::Mat} se následně zpracuje sítí a~vydá výstup, což je opět čtyřdimenzionální instance \texttt{cv::Mat}, jejíž rozměry jsou
\begin{itemize}
\item počet vstupních obrázků,
\item počet výstupních parametrů,
\item výška výstupů a
\item šířka výstupů.
\end{itemize}
Počet výstupních parametrů je 44, ale budu používat pouze prvních 15, které reprezentují části atletova těla potřebné pro vytvoření kostry. Výslednou hodnotou ve výstupu na dané pozici $(i,n,x,y)$ je pravděpodobnost, že bod na řádku $x$, ve sloupci $y$ ve výstupu odpovídá části těla $n$ ve vstupním obrázku $i$. Abych dostal pozici výstupního bodu ve vstupním obrázku, je nutné výstupní souřadnice škálovat poměrem velikostí vstupu a~výstupu sítě.

Jelikož se atlet při skoku otáčí a~model je natrénovaný na postavách, které otočené nejsou, bude pro lepší přesnost detekce potřeba otáčet také vstupní obrázek. Detekci tedy bude vhodné spouštět na více obrázcích zároveň, což siť umožňuje. Vstupy sítě vytvořím z~obrázků metodou \texttt{cv::dnn::blobFromImages}, která k~tomuto účelu slouží.

\subsection{Zisk modelu atleta z~detekované kostry}

\subsection{Analýza pohybu atletova těla}

\subsection{Zobrazení výstupu analýzy}

















