\chapter{Realizace}

\section{Návrh řešení}

\subsection{Zpracování videa}

...opencv

\subsection{Nalezení atletova těla ve videu}

Prvním krokem k~analýze videozáznamu je nalezení pozice atletova těla ve videu. Pro její určení stačí znát pozici ohraničujícího rámečku atletova těla ve snímku, v~němž je poprvé vidět. Může se stát, že v~prvním snímku videa atlet vidět není, a tedy je nutné specifikovat i~číslo snímku, ve kterém atletův ohraničující rámeček znám.

K~zisku těchto informací lze přistoupit dvěma metodami. První z~nich je manuální zadání ohraničujícího rámečku uživatelem. Pro jednoduchst práce s~programem je vhodné použít grafické rozhraní, které dovolí uživateli definovat rámeček pohybem myši po snímku videa. K~tomuto účelu se výborně hodí metoda \texttt{cv::selectROI}.

Druhou metodou nalezení atletovy pozice ve videu je automatická detekce. K~její realizaci nejprve detekuji ohraničující rámečky postav ve snímku videa a~následně budu sledovat jejich pohyb.

Pro detekci postav ve snímku lze použít strukturu \texttt{cv::HOGDescriptor} a~její metodu \texttt{detectMultiScale}, která je schopná detekovat několik postav různých velikostí v~daném snímku. K~detekci postav ve snímku využívá HOG a~SVM.

Detekované postavy je potřeba trasovat i v~následujících snímcích. K~tomu využiji třídu \texttt{cv::Tracker}.

Jelikož je mým cílem detekce atleta, stačí sledovat jen postavy, které se hýbou. Pohyb postavy určím podle vzájemného pohybu ohraničujícího rámečku postavy a~pozadí snímku. Pohyb pozadí ve videu budu trasovat stejně jako pohyb postav. Vyberu si část pozadí a~její pohyb získám s~použitím třídy \texttt{cv::Tracker}.

Atleta odliším od ostatních pohybujících se postav tak, že provede odraz, tedy se směr jeho pohybu změní směrem vzhůru. Jakmile k~odrazu dojde, zapamatuji si číslo snímku, v~němž jsem atleta poprvé detekoval, a~pozici jeho ohraničujícího rámečku v daném snímku.

\subsection{Detekce kostry atleta}



\subsection{Zisk modelu atleta z~detekované kostry}

\subsection{Analýza pohybu atletova těla}

\subsection{Zobrazení výstupu analýzy}





\section{Detekce kostry atletova těla}

Řešení detekce kostry atletova těla ve videozáznamu jsem rozdělil na tři úkoly. Prvním z~nich je nalezení pozice atletova těla ve snímku videa. Následujícím úkolem je sledování pohybu atletova těla průběhu videa a~poslední částí je detekce kostry těla v jednotlivých snímcích.

\subsection{Nalezení pozice těla}

Prvním krokem k~detekci kostry atletova těla je určení pozice těla ve snímku videa. Pozici těla budu odhadovat obdélníkem ve videu, který tělo atleta obklopuje, tedy ohraničujícím rámečkem.

Nabízí se dvě možnosti, jak pozici příslušného ohraničujícího rámečku

\subsection{Sledování pohybu těla}

\subsection{Detekce částí těla}
