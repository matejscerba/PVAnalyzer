\chapter{Realizace}
\label{chap:realizace}

\section{Zvolené prostředky}

\section{Schéma programu}



\section{tmp0}

Každému snímku také nastavím velikost, aby odpovídala rozlišení 720p. Video v~portétním režimu upravím, aby mělo šířku 720~pixelů, pokud je video natočeno na šířku, změním jeho výšku na 720~pixelů.

\section{tmp}

Ve snímku, ve kterém detekuji objekt, vyberu část snímku, ve které se objekt, jehož pohyb určuji, nenachází. Tím získám pozadí, jehož pozice je statická v~průběhu celého videa. Pozici objektu budu následně vztahovat ke středu této oblasti. Určení pozice je v~prvním snímku snadné, lze ji vyjídřit výrazem $p_0=o_0-b_0$, kde $o_0$~je pozice středu objektu ve snímku, $b_0$~pozice středu pozadí ve snímku a $p_0$~výsledná pozice objektu vůči pozadí. Také určím pozici $t_0$ levého horního rohu snímku vůči středu pozadí, ta je rovna $t_0=-b_0$. Na základě pozice levého horního rohu snímku lze snadno určit pozici objektu: $p_0=t_0+o_0=o_0-b_0$.

Při přechodu do následujícího snímku získám pomocí trackerů pozici objektu a~pozadí. Podle pozice pozadí v~nyní zpracovávaném snímku získám pozici levého horního snímku vůči pozadí, jelikož se pozadí nepohybuje, reprezentuje tato hodnota pozici levého horního rohu snímku vůči středu původního pozadí. Tuto hodnotu si označím jako $t_1$. Podle její hodnoty snadno určím novou pozici objektu, dojdu k~ní výrazem $p_1=t_1+o_1$, kde $o_1$~je pozice středu objektu v~nyní zpracovávaném snímku a $p_1$~výsledná pozice vůči původnímu středu pozadí.

V~průbehu videa pravděpodobně dojde k~momentu, kdy se pozorovaná část pozadí dostane mimo záběr. V~tuto chvíli aktualizuji pozadí jako novou oblast pozadí právě zpracovávaného snímku. Na základě vzájemné pozice původního a~aktualizovaného pozadí určím pozici aktualizovaného pozadí vůči původnímu pozadí, označím si ji jako $d$. Pozice objektů v~následujících snímcích budu vztahovat k~aktualizovanému pozadí, ale k~výsledné pozici přičtu $d$. Při další aktualizaci pozadí zvýším $d$ o~posun nového pozadí vůči stávajícímu. [OBRÁZEK znázorňující výpočty]

Tyto výpočty vycházejí z~vlastností souřadného systému snímku, který popisuji v~sekci \ref{ssec:konverze}.

\section{tmp2}

Klouby kostry reprezentuji v~souřadném systému snímku videa. Bod nacházející se na pozici $(x,y)$ ve snímku se nachází ve vzdálenosti $x$\,\rm pixelů od levé strany snímku a~ve vzdálenosti $y$\,\rm pixelů od horní strany snímku.

Pro převod kostry do 3D modelu je tedy třeba sečíst pozici levého horního rohu vůči pozadí a~pozici kloubu kostry ve snímku, abych získal pozici $(x,y)$ kloubu kostry vůči pozadí. Následně převrátím první složku, pokud atlet běží doleva, jinak ji nechám beze změny. Druhou složce výsledného bodu nastavím na $0$. Jako hodnotu poslední složky uložím hotnotu $-y$. Výsledný bod posunu o~$-d$, kde $d$ je pozice kotníku odrazové nohy při odrazu. Tuto hodnotu získám až po analýze parametrů skoku (popsané v~sekci \ref{ssec:analyza}), kterou provádím v~posunutém souřadném systému, což na parametry skoku nemá vliv.

Konverzi bodu kostry $(x',y')$ ve snímku, jehož levý horní roh má pozici $(x'',y'')$ vůči pozadí, do 3D modelu lze reprezentovat takto:
\begin{alignat*}{3}
&\text{Při běhu doprava:} &\qquad (x,y,z)&=(x'+x'',&0,-y'-y'')-d. \\
&\text{Při běhu doleva:} &\qquad (x,y,z)&=(-x'-x'',&0,-y'-y'')-d.
\end{alignat*}
$d$ značí pozici kotníku odrazové nohy při odrazu vůči pozadí.











