\chapter*{Úvod}
\addcontentsline{toc}{chapter}{Úvod}

Při trénincích a~soutěžích ve skoku o~tyči využívají atleti spolu s~trenéry videozáznamy provedených skoků. Videozáznam poté využívají pro detekci chyb v~technickém provedení skoku a~jejich případný vliv na výsledný výkon. Pro výkon při skoku o~tyči jsou kromě techniky důležité take biomechanické parametry skoku. Mezi tyto parametry patří napřáklad rychlost atleta při rozběhu.

V~tréninkovém procesu je důležité sledovat výkonnost atleta. K~tomuto účelu slouží především měření času běhu různých vzdáleností. Pro skok o~tyči je vhodné aplikovat podobné metody právě při skokanských trénincích. Jednou z~možností, jak toho docílit, je použití specializované techniky pro měření rychlosti atleta při rozběhu. Mezi specializovanou techniku lze zařadit radary nebo fotobuňky. Za použití vhodně rozmístěných fotobuněk lze poměrně přesně určit rychlost v~různých částech rozběhu. Využití těchto metod je finančně náročné a~jediným parametrem, který lze sledovat je pouze rozběhová rychlost atleta.

S~rozvojem počítačového vidění přibývá na světových atletických soutěžích studií zabývajících se pohyby sportovců \citep{IAAFBR}. Výsledky těchto studií jsou pro atlety a~jejich trenéry vynikající zpětnou vazbou. Tyto studie využívají několik kamer k~zachycení pohybu atleta, kamery nejprve kalibrují za použití přesně změřených konstrukcí s~vyznačenými body. Pro analýzu pohybu poté využívají 3D model atleta, který z~videí získají. Výsledný model zachycuje polohu kloubů těla a~na základě jejich pohybů lze jeho pohyb analyzovat. Takovýto model je možné použít k~analýze mnohem více parametrů skoku o~tyči než jen rychlosti běhu atleta.

Studie se většinou zabývají porovnáváním pohybu různých atletů na základě získaných parametrů. V~tréninkovém prostředí je zvykem využívat pouze videozáznamy a~případné porovnání skoků probíhá spuštěním záznamů, které dané skoky zachycují, následným rozborem techniky jednotlivých skoků a~hledáním případných rozdílů. Tato forma porovnání skoků ovšem nebere v~potaz biomechanické aspekty pohybu atleta, které mohou techniku skoku ovlivnit. Hodnoty parametrů, kterými se specializované studie zabývají, lze z~videa odhadnout pro účely porovnání skoků, ale ani zkušení trenéři nejsou schopni přesně určit jejich hodnoty.

Podobné studie pohybu jsou užitečným tréninkovým prostředkem, ovšem jejich využití jen v~malém počtu tréninků se nejeví jako reálné. Dostupnějším přístupem k tomuto problému by byla aplikace, běžící na mobilním zařízení, která by analyzovala pohyb atleta na základě modelu získaného z~jediného videa. Výsledkem této analýzy by byly graficky znázorněné parametry skoku a~případné porovnání s~hodnotami parametrů jiných skoků. Takováto aplikace by výrazně přispěla k~přesnější analýze tréninkových a~závodních skoků mimo soutěže světové úrovně.

Má práce představuje základ pro vznik podobné aplikace. Mým cílem je extrakce modelu atleta z~jediného videozáznamu, jeho následná analýza pro zisk hodnot příslušných parametrů a~následné zobrazení hodnot parametrů jejich zanesením do grafů.

Text jsem rozdělil do tří kapitol. V~první se zabývám teorií. V~této kapitole popisuji skok o~tyči, způsoby jeho analýzy, možnosti využití videozáznamů a~důležité biomechanické parametry skoku. Dále se ve stejné kapitole zabývám teorií počítačového vidění, konkrétně detekcí člověka, způsobem fungování trackerů a~detekcí kostry člověka.

Druhá kapitola se věnuje realizaci programu. V~úvodu této kapitoly se věnuji pracem zabývajícím se související tematikou, následně rešerši dostupných knihoven k~realizaci programu, stručnému návrhu řešení a~samotné realizaci. V~poslední zmíněné části se věnuji podrobnému popisu programu.

Ve třetí a~zároveň poslední kapitole se věnuji experimentům, konkrétně zisku testovacích dat a~rozboru funkcionality programu na různých videích.